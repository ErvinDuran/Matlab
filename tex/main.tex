\documentclass{scrartcl}

\usepackage{amsmath}
\usepackage{array}
\title{Cuadernillo de Ejercicios}
\author{Ing. Ervin Adrian Duran Aguilar}
\date{}

\begin{document}
	\maketitle
	\section{Secuenciales}
	
	\begin{enumerate}
		\item Dado un numero de 4 dígitos halle la suma de dígitos en la posición par y el producto de los dígitos en la posición impar. Par ello realice el conteo de izquierda a derecha de los dígitos.
		\item Un comerciante desea saber cual es el precio al que debe que debe promocionar sus productos los cuales venderá con factura. Para ello tiene el precio al que adquiere cada producto y al que debe adicionar el 16\% para hallar el precio de venta.
        
        \item Las poblaciones tienden a expandirse exponencialmente. Esto es 
        \begin{equation*}
        	P = P_{0}e^{rt}
        \end{equation*}
        
        donde:
		\begin{eqnarray*}
			P		   &=&	\mbox{población actual} \\
			P_{o}  &=& \mbox{Población original} \\
			r 			& = & \mbox{Tarifa de crecimiento continua, expresado como fracción}\\
			t 			& = & \mbox{tiempo}
		\end{eqnarray*}
		
		Si originalmente se tienen 100 conejos que se reproducen a una tasa de crecimiento constante de 90\% (r = 0.9) por año, encuentre cuántos conejos tendrá al final de 10
		años.
	\end{enumerate}
		
\end{document}